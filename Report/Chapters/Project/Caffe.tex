%!TEX root = ../../Main.tex
\graphicspath{{Chapters/Project/}}
%-------------------------------------------------------------------------------

\section{Caffe} % (fold)
\label{sec:caffe}

Convolutional Architecture for Fast Feature Embedding (Caffe) is a public available framework which makes it possible to design, train and use a convolutional in a rather simple way. The source code is available online and the team behind the framework encourage other people to note or change any errors which they might find. The framework allows you to choose to use either the CPU or the GPU for calculations. In this project the CPU is used.



In order to use Caffe for image classification three files are needed - a .prototxt file, a \_solver.prototxt file and a .caffemodel file. These three files together defines the architecture of the network, the hyperparameters and the training and testing strategy.

The .caffemodel file defines the architecture of the model e.g. the number and types of layers. An example of a defined layer is shown in

The .prototxt file defines ...

The \_solver.prototxt file defines ...


% section caffe (end)